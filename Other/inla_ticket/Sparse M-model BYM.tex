\documentclass{article}
\usepackage{amsmath}
\usepackage{amssymb}
\begin{document}


Let us consider a $k$-variate Besag-York-Mollié model. We assume disease-specific mixing parameters, hence we rely on the M-models \cite{MMod} framework. 

If we define the matrix-valued mixing parameter$\Phi:=\mathrm{diag}( \phi_1, \phi_2, \dots  \phi_k )$ and $\bar{\Phi}:= \mathrm{diag} ( 1 - \phi_1,  1 - \phi_2 , \dots  1 - \phi_k) = (I_k -  \Phi ) $, the BYM field $\mathbf{Y}$ is given by:

\begin{equation}
\label{eq:Mmod_conv}
\mathbf{Y} =     \mathbf{U} \Phi^{\frac{1}{2}} M  + \mathbf{V} \bar{\Phi}^{\frac{1}{2}}M
\end{equation}

Where  \begin{itemize}
\item $\mathbf{U} = (U_1, U_2, \dots U_k)$ is a multivariate and independent ICAR field, such that $U_j \sim \mathcal{N}_n(0, L^+)$ for $j = 1, 2 \dots k$, and $L^+$ is the pseudoinverse of the scaled graph Laplacian matrix
\item $\mathbf{V} = (V_1, V_2, \dots V_k)$ is a collection of $iid$ random Normal vectors, such that $V_j \sim \mathcal{N}_n(0, I_n)$ for $j = 1, 2 \dots k$.
\item $M$ is a generic $k \times k$ full-rank matrix such that $M^\top M = \Sigma$, being $\Sigma$ is the scale parameter. For instance, $M$ can be defined as $D^{1/2} E^\top$, where $D$ is the diagonal matrix of the eigenvalues of $\Sigma$ and $E$ is the corresponding eigenvectors matrix.

\end{itemize}
And the prior variance of $\mathbf{Y}$ is:
$$
\mathrm{Var}\left[ \mathrm{vec}(\mathbf{Y}) \mid \Sigma, \Phi \right] = 
\left( M^\top \otimes I_n\right) 
\mathrm{diag}\left(S_1,\ldots,S_k\right)
\left( M \otimes I_n \right)
$$
Where each diagonal block $S_j$ is the variance of $U_j \Phi + V_j \bar{\Phi}$, i.e. $S_j = \phi_j L^+ + (1-\phi_j) I_n$. 



Extending the approach of \cite{BYM2} to the multivariate case, we know that:

$$
\mathbb{E} [\mathrm{vec}(\mathbf{Y}) \mid U, \Phi , \Sigma] = \mathrm{vec}(\mathbf{U} \Phi^{\frac{1}{2}}M) =
[( M^{\top}\Phi^{\frac{1}{2}}) \otimes I_n] \mathrm{vec}(\mathbf{U})
$$

and similarly

$$
\mathrm{Var} [\mathrm{vec}(\mathbf{Y}) \mid \mathbf{U}, \Phi , \Sigma] =
\left[( M^{\top} \bar{\Phi}^{\frac{1}{2}}) \otimes I_n \right] 
\mathbb{E} \left[
\mathrm{vec}(\mathbf{V})
\mathrm{vec}(\mathbf{V})^\top \right]
\left[(\bar{\Phi}^{\frac{1}{2}} M) \otimes I_n \right] = 
\left(
 M^\top \bar{\Phi}  M \right) \otimes I_n
$$


The distribution of $\mathbf{Y} \mid \mathbf{U}, \Sigma, \Phi $ then reads:
\begin{align*}
-2 \ln \pi \left(\mathrm{vec}(\mathbf{Y}) \mid \mathbf{U}, \Sigma, \phi \right) %\\
% = C+ \Big{ \{ }
% \mathrm{vec}(\mathbf{Y}) - \left[ \left( M^\top \Phi^{\frac{1}{2}} \right) \otimes I_n \right] \mathrm{vec}(\mathbf{U}) \Big{\} } ^{\top}
%
%\left[ \left(
% M^{-1} \bar{\Phi}^{-1}{M^{-1}}^\top
%\right) \otimes I_n \right] 
%
%\Big{ \{ } \mathrm{vec}(\mathbf{Y}) -\left[ \left( M^\top \Phi^{\frac{1}{2}} \right) \otimes I_n \right] \mathrm{vec}(\mathbf{U}) \Big{\}} 
 = C + 
%
\mathrm{vec}(\mathbf{Y})^{\top}
\left[ \left(
 M^{-1} \bar{\Phi}^{-1}{M^{-1}}^\top
\right) \otimes I_n \right] 
\mathrm{vec}(\mathbf{Y}) \\ 
%
- 2\mathrm{vec}(\mathbf{Y})^{\top} 
\left[ \left( 
 M^{-1} \bar{\Phi}^{-1}{M^{-1}}^\top\right) \otimes I_n \right] 
\left[ \left( M^\top \Phi^{\frac{1}{2}} \right) \otimes I_n \right] \,\mathrm{vec}(\mathbf{U})   \\
%
+  \mathrm{vec}(\mathbf{U})^{\top}
    \left[ \left(\Phi^{\frac{1}{2}} M \right) \otimes I_n \right]
    \left[ \left(  M^{-1} \bar{\Phi}^{-1}{M^{-1}}^\top\right) \otimes I_n \right]
    \left[ (M^\top\Phi^{\frac{1}{2}} ) \otimes I_n \right]
   \mathrm{vec}(\mathbf{U}) \\
%
= C + \mathrm{vec}(\mathbf{Y})^{\top}
\left[ \left(
 M^{-1} \bar{\Phi}^{-1}{M^{-1}}^\top
\right) \otimes I_n \right] 
\mathrm{vec}(\mathbf{Y}) \\ 
%
- 2\mathrm{vec}(\mathbf{Y})^{\top} 
\left[ \left(  M^{-1} \bar{\Phi}^{-1} \Phi^{\frac{1}{2}} M^{\top} \right)
\otimes I_n \right]  \,\mathrm{vec}(\mathbf{U})   \\
%
+  \mathrm{vec}(\mathbf{U})^{\top}
    \left[ \left(  \Phi  \bar{\Phi}^{-1}  \right) \otimes I_n \right]
   \mathrm{vec}(\mathbf{U})
\end{align*}

Now, for brevity let us define the following $k \times k$ matrices:
\begin{align*}
q_{11} :=  M^{-1} \bar{\Phi}^{-1}{M^{-1}}^\top; \quad
q_{12} :=  M^{-1} \bar{\Phi}^{-1} \Phi^{\frac{1}{2}}; \quad
q_{22} := \Phi \bar{\Phi}^{-1}
\end{align*}
Hence 
\begin{align*}
-2 \ln \pi \left(\mathrm{vec}(\mathbf{Y}) \mid \mathbf{U}, \Sigma, \Phi \right) 
= C + \mathrm{vec}(\mathbf{Y})^{\top}
\left(q_{11} \otimes I_n \right) 
\mathrm{vec}(\mathbf{Y}) \\
%
- 2\mathrm{vec}(\mathbf{Y})^{\top} 
\left(  q_{12} \otimes I_n \right)  \,\mathrm{vec}(\mathbf{U})\\
%
+  \mathrm{vec}(\mathbf{U})^{\top}
    \left(q_{22} \otimes I_n \right)
   \mathrm{vec}(\mathbf{U})
\end{align*}
Then, we have 
\begin{align*}
-2 \ln \pi \left(\mathrm{vec}(\mathbf{Y}), \mathrm{vec}(\mathbf{U}) \mid \Sigma, \phi \right)  
= C + \mathrm{vec}(\mathbf{Y})^{\top}
\left(q_{11} \otimes I_n \right) 
\mathrm{vec}(\mathbf{Y})\\ 
%
- 2\mathrm{vec}(\mathbf{Y})^{\top} 
\left(  q_{12} \otimes I_n \right)  \,\mathrm{vec}(\mathbf{U})\\  
%
+  \mathrm{vec}(\mathbf{U})^{\top}
    \left(q_{12} \otimes I_n + I_k \otimes L\right)
   \mathrm{vec}(\mathbf{U})
\end{align*}
Hence, with some straightforward algebra, it can be concluded that:
\begin{equation} \label{eq:joint_bym_mmod}
    \begin{pmatrix}
        \mathrm{vec} (\mathbf{Y}) \\ \mathrm{vec} (\mathbf{U}) 
    \end{pmatrix}
    \sim N_{2kn} \left( 0, \begin{pmatrix}
            q_{11} \otimes I_n \, & 
            \, - q_{12} \otimes I_n \\
            - q_{12}^\top\otimes I_n \, & \,
             q_{22} \otimes I_n + I_k \otimes L
        \end{pmatrix}^{-1} \right)
\end{equation}
Which generalises to the multivariate case the sparse precision derived by \cite{BYM2}. 




\begin{thebibliography}{6}
\bibitem{MMod}
P. Botella-Rocamora, M.A. Martinez-Beneito, and S. Banerjee. A unifying
modeling framework for highly multivariate disease mapping. Statistics in
Medicine, 34(9):1548–1559, 2015

\bibitem{BYM2}
A. Riebler, S. H. Sørbye, D. Simpson, and H. Rue. An intuitive Bayesian spa-
tial model for disease mapping that accounts for scaling. Statistical methods
in medical research, 25(4):1145–1165, 2016

\end{thebibliography}



\end{document}